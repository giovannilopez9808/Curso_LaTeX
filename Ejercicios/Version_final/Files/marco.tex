\section{Marco teórico}
Se considera un cilindro uniforme de radio R que rueda sin deslizarse sobre una superficie horizontal.
Conforme el cilindro da vueltas a través de un ángulo theta, su centro de masa se mueve una distancia
lineal $s=R\theta$ . Por lo tanto, la rapidez traslacional del centro de masa para movimiento de rodamiento
puro está dado por:
\begin{equation}
    v_{cm}=\frac{ds}{dt}=R\frac{d\theta}{dt}=R\omega,
    \label{eq:vcm}
\end{equation}
donde $\omega$ es la rapidez angular del cilindro. La ec. \{\ref{eq:vcm}\} se cumple siempre que un cilindro o esfera ruede
sin deslizarse y es la \textit{condición para movimiento de rodamiento puro}. La magnitud de la aceleración
lineal del centro de masa para movimiento de rodamiento puro es
\begin{equation}
    a_{cm}=\frac{dv_{cm}}{dt}=R\frac{d\omega}{dt} = R\alpha,
    \label{eq:acm}
\end{equation}
donde $\alpha$ es la aceleración angular del cilindro. Permaneciendo en un marco de referencia en reposo respecto a CM, el cuerpo se observará en rotación pura alrededor de CM. En ese caso, cada punto del objeto
tiene la misma velocidad angular $\omega$. Por lo tanto, la velocidad particular de cada punto viene dada por a
ec. \{\ref{eq:vcm}\}, donde R varía en función de la distancia del punto con respecto a CM. Para el caso en el que el
objeto no gira, sino que tiene una traslación pura, todos los puntos del objeto se mueven a velocidad
constante $v$ CM respecto a un marco de referencia externo. Resulta que en un marco de referencia externo
la velocidad instantánea $v$ de cualquier punto del objeto bajo la condición de rodamiento puro viene dada
por la suma de las velocidades del mismo punto bajo los casos de rotación pura y traslación pura.
\begin{align*}
    v_p=R\omega -v_{cm} &=v_{cm}-v_{cm}=0,\\
    v_{cm} &=0+v_{cm}=v_{cm},\\
    v_{q}&= R\omega +v_{cm}=2v_{cm}.
\end{align*}
Ya que el punto P tiene velocidad traslacional cero en cualquier instante dado, el objeto rueda en exac-
tamente la misma forma que si la superficie se retirara y el objeto girara en torno a un eje que pasa a
través de P. La energía cinética total de este objeto que se piensa que está girando se expresa como:
\begin{equation}
    K=\frac{1}{2} I_p \omega^2
    \label{eq:k}
\end{equation}
donde I\textsubscript{P} es el momento de inercia en torno a un eje de rotación a través de P. Siendo así, este modelo
aplica de forma exactamente igual al verdadero objeto bajo movimiento de rodamiento puro. Aplicando
el teorema de ejes paralelos.
\begin{equation}
    I_P= I_{CM} + MR^2,
    \label{eq:IP}
\end{equation}
sustituyendo la ecuación \ref{eq:k} en \ref{eq:IP} se obtiene que:
\begin{equation*}
    K= \frac{1}{2}I_{CM} \omega^2+\frac{1}{2}Mv_{CM}^2.
\end{equation*}
Se puede usar análisis de energía para tratar una clase de problemas concernientes al movimiento de
rodamiento de un objeto sobre un plano inclinado rugoso. Para este caso, el movimiento de rodamiento
acelerado sólo es posible si una fuerza de fricción está presente entre la esfera y el plano para producir
un momento de torsión neto en torno al centro de masa. A pesar de la presencia de fricción, no se presenta
pérdida de energía mecánica porque el punto P está en reposo en relación con la superficie en cualquier
instante. Siendo $K = mgh$,
\begin{equation*}
    mgh=\frac{1}{2} I \omega^2 \qquad I=\frac{2}{5}mR^2+mb^2,
\end{equation*}
\begin{equation}
    mgh= \frac{1}{2}\left(\frac{2}{5}mR^2+mb^2 \right)\omega^2,
    \label{eq:mgh}
\end{equation}
\begin{align*}
    \omega^2 &= \left(\frac{v}{b}\right)^2,\\
    x=\left(\frac{v_0+v}{2}\right)t &\qquad v_0=0,\\
    x=\frac{v}{2}t &\Leftrightarrow  v=\frac{2x}{t},
\end{align*}
\begin{equation}
    \omega^2=\frac{4x^2}{b^2t^2}.
    \label{eq:omega2}
\end{equation}
Sustituyendo la ecuación \ref{eq:omega2} y \ref{eq:mgh} se tiene que:
\begin{align*}
    mgh&= \frac{1}{2}\left(\frac{2}{5}mR^2+mb^2 \right)\frac{4x^2}{b^2t^2},\\
    h & = \frac{4x^2}{2mg}\left(\frac{2mR^2}{5b^2}+\frac{mb^2}{b^2}\right),\\
\end{align*}
\begin{equation}
    h=\frac{2x^2}{g}\left(\frac{2R^2}{5b^2}+1 \right)t^{-2},
    \label{eq:h}
\end{equation}
donde
\begin{equation*}
    b^2=R^2-\left(\frac{a}{2}\right)^2
\end{equation*}