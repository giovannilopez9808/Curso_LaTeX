\section{Marco teórico}
Se considera un cilindro uniforme de radio R que rueda sin deslizarse sobre una superficie horizontal.
Conforme el cilindro da vueltas a través de un ángulo theta, su centro de masa se mueve una distancia
lineal $s=R\theta$ . Por lo tanto, la rapidez traslacional del centro de masa para movimiento de rodamiento
puro está dado por:
\begin{equation}
    v_{cm}=\frac{ds}{dt}=R\frac{d\theta}{dt}=R\omega
    \label{eq:vcm}
\end{equation}
donde $\omega$ es la rapidez angular del cilindro. La ec. \{\ref{eq:vcm}\} se cumple siempre que un cilindro o esfera ruede
sin deslizarse y es la \textit{condición para movimiento de rodamiento puro}. La magnitud de la aceleración
lineal del centro de masa para movimiento de rodamiento puro es
\begin{equation}
    a_{cm}=\frac{dv_{cm}}{dt}=R\frac{d\omega}{dt} = R\alpha
    \label{eq:acm}
\end{equation}
donde $\alpha$ es la aceleración angular del cilindro. Permaneciendo en un marco de referencia en reposo respecto a CM, el cuerpo se observará en rotación pura alrededor de CM. En ese caso, cada punto del objeto
tiene la misma velocidad angular $\omega$. Por lo tanto, la velocidad particular de cada punto viene dada por a
ec. \{\ref{eq:vcm}\}, donde R varía en función de la distancia del punto con respecto a CM. Para el caso en el que el
objeto no gira, sino que tiene una traslación pura, todos los puntos del objeto se mueven a velocidad
constante $v$ CM respecto a un marco de referencia externo. Resulta que en un marco de referencia externo
la velocidad instantánea $v$ de cualquier punto del objeto bajo la condición de rodamiento puro viene dada
por la suma de las velocidades del mismo punto bajo los casos de rotación pura y traslación pura.
\begin{align*}
    v_p=R\omega -v_{cm} &=v_{cm}-v_{cm}=0\\
    v_{cm} &=0+v_{cm}=v_{cm}\\
    v_{q}&= R\omega +v_{cm}=2v_{cm}
\end{align*}